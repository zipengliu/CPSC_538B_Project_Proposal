\section{Proposed Approach}
\label{sec:proposed-approach}

Out goal is to visualize distributed state in a comprehensive way to
provide developers with an intuition about faulty behaviour. In order
to capture distributed state through variable logging, and the
aggregation of individual node state, we will leverage \dinv, a tool
for detecting likely data invariants in distributed
systems~\cite{dinv}.

Distributed state is inherently complex. In its raw form it consists
of partially ordered instances of variable values. This raw data is
classically hard to reason about. In order to convey distributed state
to a developer it must be summarized. We propose that the behaviour of
a distributed system under correct execution follows predicable
patters. Further that these patters are reflected the systems state.
The first contribution that our project seeks to make is a state
transition function $diff$ which measures the difference between
instances of distributed state. We recognize that many functions for
measuring state transitions could exist, our goal is to identify a
$diff$ function which is sensitive to state transitions, and is
normalized during repetitious behaviour.

Our initial approach will be to measure state transitions as the XOR
difference between variables in separate instances of distributed
state. We formalize the state $\sigma$ of a system to be a vector of
variables $V = {v_1,v_2,\dots,v_n}$. The XOR difference between two
states $\sigma_i$ and $\sigma_j$ is $XOR(V_i,V_j)$. The resulting
vector contains an indices for the XOR difference between each
variable. The vector itself is $n$ dimensional and represents the XOR
velocity of the systems state. To measure the effectiveness of our
$diff$ function, we will initially plot its output as a simple line
graph and check if predictable repetitions patters emerge between
executions.  Furthermore, we plan to plot the first and second order
derivatives of $diff$, $diff'$ and $diff''$ respectively to determine
if they also exhibit predictable patters. Initially we will try to
detect patterns in Ricart-Agrawala. If predictable patters emerge we
introduce a bug which causes the algorithm to violate mutual
exclusion. If predictable patterns emerge we will expand our research
into larger more general system. If the difference is not noticeable
but patterns emerge we will attempt to revise our visualization to
better reflect the property. If no alternative visualizations
demonstrate the irregularity, we will refine our $diff$ function to be
more specific to Ricart-Agrawala's state variables.

To visualize the transitional states, we need to first abstract the 
data and task, according to Munzner's nested model ~\cite{munzner2009nested}.
The original raw data to visualize can be regarded as a two dimensional 
spreadsheet, where each row represent a distributed state (variable),
 each column a snapshot and each cell the value of the variable at
 that particular snapshot.  The data type of the value is flexible,
 be it number, string, array or complicated dictionary.
However, there is little to do with this raw data if we do not derive
simplified data or transform it.  The way to do it actually depends on 
the transitional function we will experiment and concrete questions we
would like to answer with the visualization.  For example, if users want
to know the degree of change between snapshots, we can use the XOR function
to reduce each column (a vector of states at a snapshot) to a number,
whose difference is easily perceived when shown in a line chart.
It is yet early to talk about how kind of visual encoding to use before
we have a better understanding of the data and task.